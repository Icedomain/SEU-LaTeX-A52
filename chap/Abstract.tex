% !TeX root = ../main.tex
% !TEX TS-program = xelatex
% !TeX encoding = UTF-8


\begin{abstract}{希腊字母,腓尼基字母,语言,深度学习}
希腊字母源自腓尼基字母。腓尼基字母只有辅音,从右向左写。希腊语的元音发达,希腊人增添了元音字母。因为希腊人的书写工具是蜡板,有时前一行从右向左写完后顺势就从左向右写,变成所谓“耕地”式书写,后来逐渐演变成全部从左向右写。字母的方向也颠倒了。罗马人引进希腊字母,略微改变变为拉丁字母,在世界广为流行。希腊字母广泛应用到学术领域,如数学等。

希腊字母是希腊语所使用的字母,是世界上最早的有元音的字母,也广泛使用于数学、物理、生物、天文等学科。俄语等使用的西里尔字母也是由希腊字母演变而成。希腊字母进入了许多语言的词汇中,英语单字“alphabet”(字母表),源自拉丁语“alphabetum”,源自希腊语“αλφαβητον”,即为前两个希腊字母α(“Alpha”)及β(“Beta”)所合成,三角洲(“Delta”)这个词就来自希腊字母Δ,因为Δ是三角形。
\end{abstract}

\begin{englishabstract}{Greek Alphabet, Phoenician Alphabet, Language, Deep Learning}
The Greek alphabet has been used to write the Greek language since the late 9th century BC or early 8th century BC It was derived from the earlier Phoenician alphabet, and was the first alphabetic script to have distinct letters for vowels as well as consonants. It is the ancestor of the Latin and Cyrillic scripts.Apart from its use in writing the Greek language, in both its ancient and its modern forms, the Greek alphabet today also serves as a source of technical symbols and labels in many domains of mathematics, science and other fields.

In its classical and modern forms, the alphabet has 24 letters, ordered from alpha to omega. Like Latin and Cyrillic, Greek originally had only a single form of each letter; it developed the letter case distinction between upper-case and lower-case forms in parallel with Latin during the modern era.
\end{englishabstract}



